Procedure for PCSVN estimation

For each sample:

\begin{quote}
\begin{enumerate}
\def\labelenumi{\Alph{enumi})}
\item
  \begin{description}
  \item[Preprocessing]
  \begin{enumerate}
  \def\labelenumii{\arabic{enumii})}
  \item
    \begin{description}
    \item[RGB to single channel]
    + Luminance transform . Isolate green channel (Almoussa, Huynh)
    \end{description}
  \item
    \begin{description}
    \item[Remove glare]
    \begin{enumerate}
    \def\labelenumiii{\alph{enumiii})}
    \item
      \begin{description}
      \item[Mask glare]
      + Threshold (\emph{175/255}) . Threshold at \emph{80\%} of max
      intensity (Almoussa) . Lange (2005) (multistep procedure, done in
      RGB space actually) . Lamprinou (2018) (good overview here)
      \end{description}
    \item
      \begin{description}
      \item[Post-process mask]
      + Dilate with radius \emph{2} . Do nothing
      \end{description}
    \item
      \begin{description}
      \item[Inpaint glare]
      + Hybrid inpainting, with size threshold \emph{32} . Biharmonic
      inpainting . Mean value of boundary . Median value of boundary .
      Windowed mean (radius: \emph{15})
      \end{description}
    \end{enumerate}
    \end{description}
  \end{enumerate}
  \end{description}
\end{enumerate}

\begin{description}
\item[B) Multiscale Frangi filter]
\begin{enumerate}
\def\labelenumi{\arabic{enumi})}
\item
  \begin{description}
  \item[Define parameters]
  \begin{enumerate}
  \def\labelenumii{\alph{enumii})}
  \item
    \begin{description}
    \item[Scales \{σ\}]
    = n\_scales (default: \emph{40}) = scale\_range (default
    \emph{{[}-2, 3.5{]}}) = scale\_type (\emph{logarithmic base 2} or
    linear or custom) -\textgreater{} build scales
    \end{description}
  \item
    \begin{description}
    \item[Betas \{β\}]
    = \emph{0.5} each scale or custom range
    \end{description}
  \item
    \begin{description}
    \item[Gammas \{γ\}]
    \begin{description}
    \item[= strategy: (half L2 hessian norm or \emph{half hessian
    frobenius norm})]
    or custom value each scale
    \end{description}

    = redilate plate per scale (?)
    \end{description}
  \item
    \begin{description}
    \item[Dilate per scale]
    + Custom function of scale (default \emph{max\{10, int(4σ)\} if (σ
    \textless{} 20) else int(2σ))}) . No dilation
    \end{description}
  \item
    \begin{description}
    \item[Scale space convolution method]
    + Discrete Gaussian kernel with FFT . Sampled gaussian kernel with
    FFT . Sample gaussian kernel, standard convolution
    \end{description}
  \end{enumerate}
  \end{description}
\item
  \begin{description}
  \item[For σ in Σ: do Uniscale Frangi Filter]
  \begin{enumerate}
  \def\labelenumii{\alph{enumii})}
  \item
    gauss blur image with method from (1e)
  \item
    \begin{description}
    \item[take gradient across each axis, take gradient across each axis
    of gradient to get]
    Hxx, Hxy, Hyy
    \end{description}
  \item
    find eigenvalues of hessian at each point (using np.eig) and sort by
    magnitude
  \item
    zero out principal directions according to Dilate Per Scale
  \item
    \emph{zero out hessian according to max(ceil(σ),10) INSTEAD} LOOK AT
    THIS
  \item
    Calculate Frangi Vesselness Measure
  \end{enumerate}
  \end{description}
\end{enumerate}
\item[C) Merging Frangi scores]
\begin{enumerate}
\def\labelenumi{\arabic{enumi})}
\tightlist
\item
  Calculate Fmax and Fmax.where -\textgreater{} Fmax
\item
  Threshold at 95th percentile -\textgreater{} approx
\end{enumerate}

3) Compare to Trace etc.
\end{description}
\end{quote}
